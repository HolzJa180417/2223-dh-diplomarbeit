\documentclass[letterpaper,10pt]{article}
\usepackage[backend=biber, sorting=none]{biblatex}
\addbibresource{references.bib}
\title{Arbeitsprotokolle der Diplomarbeit}
\date{20.06.2022}
\author{Jonas Denk}
\begin{document}
\maketitle
\frenchspacing
\raggedbottom
\pagestyle{plain}
\section{15.06.2022 - Besprechung mit BizzNetIT}
Heute sind ich und Jakob zur Firma gefahren, um zu besprechen, welchen Umfang die App haben soll und wie dies technisch realisiert werden soll. Anschließend haben wir ein Dokument für Haslinger Peter aufgesetzt, da wir diesen gerne als Diplomarbeitsbetreuer hätten. Wir werden in den kommenden Wochen eine Besprechung mit der Firma, zwecks Datenbankaufbau, haben. Wir bekommen von Ihnen einen Zugang auf ihren Datenbankserver, damit wir dort eine Datenbank aufbauen und befüllen können. Ich habe ebenfalls ein GitHub-Repository erstellt. Ich war in der Firma von 12:45 bis 16:15, die Besprechung hat von 14:00 - 15:45 gedauert.
\section{20.06.2022 - Beginn mit Testung der GoogleMapsAPI}
Ich habe heute ein neues Flutter Projekt erstellt zum seperaten Testen der GoogleMapsAPI. Dafür erstellte ich mit meinem Chef den API-Key und recherchierte über das Thema. Anschließend probierte ich einiges aus. Die App kann jetzt zwei Positionen festlegen: Origin und Destination und navigiert quasi von A nach B. Zumindest wird der Weg eingezeichnet. Zusätzlich wird noch die Distanz in km und die umgefähre Fahrdauer angezeigt. Ausserdem schaute ich mir an, wie ich eine Flutter-App im Release-Mode auf das iPhone laden kann, damit das iPhone nicht über Kabel am Mac angeschlossen bleiben muss während des Testen. Ausprobiert wurde die App erstmal nur auf iOS, sollte auf Android jedoch genauso funktionieren. Dauer: 13:00 - 17:00
\printbibliography
\end{document}
